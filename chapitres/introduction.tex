\chapter*{Introduction}

Je peux citer des extraits de documents. J'ai par exemple reproduit
ci-dessous, par extraits, le discours de Condorcet, prononcé à
l'assemblée nationale le 12 juin 1790, au nom de l'Académie des
Sciences.

Voici tout d'abord un extrait \og hors-texte\fg{}\footnote{Le numéros
  de page sont fictifs.}.
%
\begin{displayquote}[p.~27]
  Messieurs,

  Vous avez daigné nous associer en quelque sorte à vos nobles
  travaux; et, en nous permettant de concourir au succès de vos vues
  bienfaisantes, vous avez montré que les sages représentants d'une
  nation éclairée ne pouvaient méconnaître ni le prix des sciences, ni
  l'utilité des compagnies occupées d'en accélérer le progrès et d'en
  multiplier l'application.

  Depuis son institution, l'Académie a toujours saisi et même
  recherché les occasions d'employer pour le bien des hommes, les
  connaissances acquises par la méditation, ou par l'étude de la
  nature: c'est dans son sein qu'un étranger
  illustre\footnote{Huygens}, à qui une théorie profonde avait révélé
  le moyen d'obtenir une unité de longueur naturelle et invariable,
  forma le premier le plan d'y rapporter toutes les mesures pour les
  rendre par là uniformes et inaltérables.
\end{displayquote}
%

Ensuite, je peux citer de longs extraits au cours d'une de mes
phrases: \textquote[p.~27]{L'Académie s'est toujours honorée dans ses
  annales d'un préjugé détruit, d'un établissement public
  perfectionné, d'un procédé économique ou salutaire introduit dans
  les arts, que d'une découverte difficile ou brillante; et son zèle,
  encouragé par votre confiance, va doubler d'activité et de
  force.}. Il est bien entendu possible de citer de courts extraits:
Condorcet insiste sur \enquote{la loi de la nature[, qui] a voulu que
  l'homme fût éclairé pour qu'il pût être juste, et libre pour qu'il
  pût être heureux} de façon à ce que les membres de l'assemblée
nationale n'oublient pas l'importance de l'instruction.

Un petit aparté: il est possible de citer des citations. Ainsi, le
zygomaticien s'exclama: \enquote{Pierre Dac ne disait-il pas:
  \enquote{Je suis pour tout ce qui est contre et contre tout ce qui
    est pour!}? Si, si, il le disait\ldots{}}, ce qui ne nous surprît
guère.

Et maintenant la suite (et fin) du discours.
%
\begin{displayquote}[p.~27]
  Et comment pourrions nous oublier jamais que les premiers honneurs
  publics, décernés par vous, l'ont été à la mémoire d'un de nos
  confrères? Ne nous est-il permis de croire que les sciences ont eu
  droit aussi quelque part à ces marques glorieuses de votre estime
  pour un sage qui, célèbre dans les deux mondes par de grandes
  découvertes, n'a jamais chéri dans l'éclat de sa renommée que le
  moyen d'appeler ses concitoyens à l'indépendance d'une voix plus
  imposante, et de rallier en Europe, à une si noble cause, tout ce
  que son génie lui avait mérité de disciples et d'admirateurs?

  Chacun de nous, comme homme, comme citoyen, vous doit une éternelle
  reconnaissance pour le bienfait d'une constitution égale et libre,
  bienfait dont aucune grande nation de l'Europe n'avait encore joui;
  et pour celui de cette déclaration des droits, qui, enchaînant les
  législateurs eux-mêmes par les principes de la justesse universelle,
  rend l'homme indépendant de l'homme, et ne soumet sa volonté qu'à
  l'empire de sa raison.  Mais des citoyens voués par état à la
  recherche de la vérité, instruits par l'expérience, et ce que
  peuvent les lumières pour la félicité générale, et de tout ce que
  les préjugés y opposent d'obstacles, en égarant ou en dégradant les
  esprits, doivent porter plus loin leurs regards, et, sans doute, ont
  le droit de vous remercier au nom de l'humanité, comme au nom de la
  patrie.

  Ils sentent combien, en ordonnant que les hommes ne seraient plus
  rien par des qualités étrangères, et tout par leurs par leurs
  qualités personnelles, vous avez assuré le progrès de l'espèce
  humaine, puisque vous avez forcé l'ambition et la vanité même à ne
  plus attendre les distinctions ou le pouvoir que du talent et des
  lumières; puisque le soin de fortifier sa raison, de cultiver son
  esprit, d'étendre ses connaissances, est devenu le seul moyen
  d'obtenir une considération indépendante et une supériorité réelle.

  Ils savent que vous n'avez pas moins fait pour le bonheur des
  générations futures, en rétablissant l'esprit humain dans son
  indépendance naturelle, que pour celui de la génération présente, en
  mettant les propriétés et la vie des hommes à l'abri des attentats
  du despotisme.

  Ils voient, dans les commissions dont vous les avez chargé, avec
  quelle profondeur de vues vous avez voulu simplifier toutes les
  opérations nécessaires dans les conventions, dans les échanges, dans
  les actions de la vie commune, de peur que l'ignorance ne rendit
  esclave celui que vous aviez déclaré libre, et ne réduisit l'égalité
  prononcée par vos lois à n'être jamais qu'un vain nom.

  Pourraient-ils enfin ne pas apercevoir qu'en établissant pour la
  première fois, le système entier de la société sur des bases
  immuables de la vérité et de la justice, en attachant ainsi par une
  chaîne éternelle les progrès de l'art social au progrès de la
  raison, vous avez étendu vos bienfaits à tous les pays, à tous les
  siècles, et dévoué toutes les erreurs, comme toutes les tyrannies à
  une destruction rapide?

  Ainsi, grâce à la générosité, à la pureté de vos principes, la
  force, l'avarice, ou la séduction, cesseront bientôt de contrarier,
  par des institutions arbitraires, la loi de la nature, qui a voulu
  que l'homme fût éclairé pour qu'il pût être juste, et libre pour
  qu'il pût être heureux. Ainsi, vous jouirez à la fois et du bien que
  vous faites, et du bien que vous préparez, et vous achèverez votre
  ouvrage au milieu des bénédictions de la foule des opprimés dont
  vous avez brisé les fers, et des acclamations des hommes éclairés
  dont vous avez surpassé les espérances.
\end{displayquote}
%

%%% Local Variables: 
%%% mode: latex
%%% TeX-master: "../book"
%%% End:
